%%%%%%%%%%%%%%%%%%%%%%%%%%%%%%%%%%%%%%%%%%%%%%%%%%%%%%%%%%%%%%%%%%%%%%%%%%%%%%%%
% TODO
%	enlever le titre mis en commentaire? ^^
%	définir l'espace entre paragraphes et éviter l'artifice du saut de ligne
%	utiliser la commande "\and" dans la liste des auteurs?
%	définir un guide de style pour la présentation du document
%	pourquoi ne pas utiliser directement les accents dans le titre?
%	mettre le document en cours de rédaction (draft - partie include)
%	distinction explicite micro-processeur central/graphique
%	pour chaque composant (de réalisation) externe (modules urilisés), définir
%		quelle spécification en question est respectée, et quelle implémentation
%		(par exemple pour les SDK)
%%%%%%%%%%%%%%%%%%%%%%%%%%%%%%%%%%%%%%%%%%%%%%%%%%%%%%%%%%%%%%%%%%%%%%%%%%%%%%%%

\author{Paul ADENOT -- Yannick MEINE -- Gael MOTTE \\ Armand ROSSIUS -- Nicolas SILVA }
%\title{Références croisées sur un ensemble de fichiers en \texttt{C++}}
\title{D\'{e}veloppement d'un effet de r\'{e}verb\'{e}ration par convolution\\Cahier des charges}
\documentclass[a4paper, 11pt]{article}



% Inclusion des paquets --------------------------------------------------------

% Réduction manuelle des marges
	\usepackage[top=20mm, bottom=20mm, left=35mm, right=35mm]{geometry}

 % Encodage du document source
	\usepackage[utf8]{inputenc}

% Gestion des caratères accentués du document source.
	\usepackage[T1]{fontenc}

% Fontes
% Helvetica (décommenter les deux lignes suivantes)
%\usepackage[scaled]{helvet}
%\renewcommand*\familydefault{\sfdefault}
% Serif

\usepackage[sc]{mathpazo}
%\usepackage{concmath}
\usepackage[francais]{babel}

% Gestion du listing de code
	\usepackage{listings}
	% Mise en place de la coloration syntaxique
		\usepackage{xcolor}
% Pour insérer des algorithmes en pseudocode
	\usepackage{algorithmic}



% Gestion de l'interlignage
	\usepackage{setspace}



% Gestion de l'index (table de signets)
	\usepackage{makeidx}
% Pour gérer les liens internes et les URL cliquables
	\usepackage{url}
% Pour mettre les liens cliquables
	\usepackage{hyperref}



% Pour les en-têtes et pieds-de-page
	\usepackage{fancyhdr}
% Pour le logo en haut a droite
	\usepackage{eso-pic}
% Pour la couverture en PDF pleine page
	\usepackage{pdfpages}
% Pour la biblio bibtex
	\usepackage{bibunits}



% Gestion de l'insertion d'images
	\usepackage{graphicx}
% Pour l'enroulement du texte autour des figures
	\usepackage{wrapfig}
% Pour gérer les éléments flottants
	\usepackage{float}
% Pour les cadres à ombrage du glossaire
	\usepackage{fancybox}
% Pour faire des sous figures correctement numérotés
	\usepackage{subfigure}



% Définition du style ----------------------------------------------------------

% Pour enlever les espaces automatiques avant les : (utile pour les namespaces)
	\NoAutoSpaceBeforeFDP
% Changer l'espace entre les paragraphes
	\parskip=1ex
% Taille du header
	\setlength{\headheight}{14.5pt}



% Permet l'ajout de code par insertion du fichier
% $2 : nom du fichier à inclure
% $3 : le type de langage (C++, C, Java ...)

\newcommand{\addCode}[2]{%

% Configuration de la coloration syntaxique du code
	\definecolor{colKeys}{rgb}{0,0,1}
	\definecolor{colIdentifier}{rgb}{0,0,0}
	\definecolor{colComments}{rgb}{0,0.5,1}
	\definecolor{colString}{rgb}{0.6,0.1,0.1}
	
% Configuration des options 
	\lstset{%
		language = #2,%
		identifierstyle=\color{colIdentifier},%
		basicstyle=\ttfamily\scriptsize, %
		keywordstyle=\color{colKeys},%
		stringstyle=\color{colString},%
		commentstyle=\color{colComments},%
		columns = flexible,%
		%tabsize = 8,%
		showspaces = false,%
		numbers = left, numberstyle=\tiny,%
		frame = single,frameround=tttt,%
		breaklines = true, breakautoindent = true,%
		captionpos = b,%
		xrightmargin = 10mm, xleftmargin = 15mm, framexleftmargin = 7mm,%
	}%
	\begin{center}
	\lstinputlisting{#1}
	\end{center}
}

% Manuel d'utilisation
\newcommand{\addExample}[6]
{
	\begin{center}
		\footnotesize
		\definecolor{lightlgray}{rgb}{0.9,0.9,0.9}
		\fcolorbox{black}{lightlgray}{
			\begin{minipage}{0.9\linewidth}
				\setlength{\fboxrule}{1pt}
				\framebox{{\Large\texttt{#1}}}
				\setlength{\fboxrule}{0.4pt}
				\footnotesize
				\begin{description}
					\item[Syntaxe] ~\\
					\texttt{#2}
					\item[Description]
					#3
					\item[Valeur de retour]~\\
					#4
					\item[Exemple d'utilisation] ~\\
					\begin{minipage}{0.9\linewidth}
						\addCode{#5}{#6}
					\end{minipage}
				\end{description} 
			\end{minipage}
		} %end fcolorbox
	\end{center}
	\normalsize
}


% Couleurs des URL et des liens internes
	\hypersetup{urlcolor=blue,linkcolor=black,citecolor=black,colorlinks=true}

% En-têtes, pieds-de-page
	\pagestyle{fancy}
	\rhead{INSA de Lyon -- 2009/2010}
	\lhead{Fil Rouge}
	\cfoot{\thepage}
	\renewcommand{\footrulewidth}{0.4pt}

\begin{document}
\maketitle
\tableofcontents
\section{Le projet}
Dans le cadre du projet «Fil rouge» proposé par les enseignants du département informatique de l'INSA de Lyon, nous nous proposons de développer un effet de réverbération par convolution.

~\\
% réponse impulsionnelle : signal sortie d'un système en réaction à une signal impulsionnel ; ici nos signaux sont des ondes sonores
% une impulsion est mathématiquement parlant une onde sans harmonique et très brès - point à développer
% la sortie doit ici représenter la configuration acoustique de la pièce
% un fichier .WAV ne signifie rien : il exite différents formats de codage audionumérique regroupés autour de l'extension wav.
% Préciser simplement : "sous forme numérique" - c'est ce qui nous importe ici, plutôt que "sous la forme d'un fichier .WAV)
%
% \subsection{principe}
La réverbération par convolution utilise le procédé mathématique du même nom pour produire, à partir d'une réponse impulsionnelle (sous la forme d'un fichier .WAV, et enregistré à partir de l'acoustique d'un lieu ou de la réponse d'un équipement audio) et d'une piste audio (aussi sous la forme d'un fichier .WAV) un signal de sortie correspondant au second signal «mélangé» au premier, de sorte que la seconde piste audio (souvent le son d'un instrument) apparaise traitée par l'acoustique du lieu ou l'effet de l'équipement audio. Ce procédé permet donc de s'affranchir du traitement acoustique du lieu d'enregistrement ou de l'achat d'équipement dédié, souvent couteux.

~\\
% définition d'un plugin, de la MAO, d'un SDK et d'un VST.
% quelle spécification du langage et sous quelle implémentation?
% Références : Steinberg, VST
% Licence d'utilisation du SDK VST 2.4
%
% \subsection{Architecture logicielle}
Le logiciel développé aura la forme d'un plugin, écrit en {\tt C++}, sera compatible avec la majorité des logiciels de musique assistée par ordinateur (MAO), parce qu'utilisant le Software Development Kit (SDK) Virtual Studio Technology (VST), créé par Steinberg, célèbre entreprise dans le domaine de l'audionumérique. Ce SDK est \emph{de facto} un standard dans le monde de la MAO, et son utilisation est gratuite.

~\\
% \subsection{Convolution}
La convolution étant une opération coûteuse au point de vue temps CPU, nous nous proposons d'effectuer le calcul sur le processeur graphique de la machine (le GPU), qui est particulièrement optimisé pour le calcul en virgule flottante massivement parallélisé, ce qui correspond tout à fait à la convolution. Cela permettra d'alléger le CPU lors de l'utilisation de multiple instances du plugin dans le programme hôte, et de proposer une latence imperceptible lors de l'enregistrement et du jeu en direct dans l'interface audionumérique de la machine.

~\\
Cinq personnes feront partie du projet, se répartissant les tâches en fonction de leurs affinités avec les technologies utilisées.
Le projet sera libre, et placé sous licence GPLv2.
\section{Technologies}
\subsection{VST SDK}
Le VST SDK est un ensemble de classes {\tt C++} reposant sur une interface de programmation (API) sous-jacente écrite en C. Cela permettra un développement aisé, puisqu'orienté objet. De plus, ce SDK fournie une abstraction matérielle complète, ainsi qu'une abstraction logicielle, puisque son fonctionnement est garanti avec bon nombre de logiciels hôtes. Nous pourrons alors nous concentrer sur l'écriture des algorithmes, tout en étant sûr du fonctionnement du logiciel en dehors de notre environnement de test.

~\\
Le plugin aura besoin d'une interface graphique conviviale, qui devra être développée, le SDK ne proposant que des primitives pour des \emph{widgets} basiques tels que des boutons ou des potentiomètres. Nous utiliserons pour cela la bibliothèque libre (sous les termes de la BSD License) VSTGUI (\url{http://sourceforge.net/projects/vstgui/}), qui propose une interface de développement d'interface graphique de plus haut niveau.
\subsection{Calcul sur GPU}
Le logiciel utilisera une solution logicielle permettant de réaliser le calcul sur le GPU, et donc de diminuer la charge du processeur principal.
À l'écriture de ces lignes, nous devons encore choisir entre deux solutions, OpenCL du Khronos Group ou CUDA de nVidia.
\paragraph{OpenCL}

\paragraph{CUDA}
% Références : CUDA, spécification 8 series, ncc
% Qu'est-ce que CUDA? Framework, bibliothèque, API?
(Compute Unified Device Architecture) est disponible sur les cartes graphiques nVidia Geforce 8 series et postérieures. Cela veut dire qu'il faut obligatoirement une carte graphique satisfaisant cette spécification pour que le calcul soit possible. Cette technologie est mature, puisque la version 2 est considérée comme stable et la version 3 disponible en beta.
~\\
CUDA est un ajout au langage C, proposant quelques fonctions pour travailler sur la carte graphique, et quelques ajouts syntaxique afin de contrôler au mieux la gestion du calcul en parallèle.
~\\
C'est une technologie propriétaire, les sources du compilateur {\tt ncc} n'étant pas disponibles.
\section{Cahier des charges}
\end{document}

